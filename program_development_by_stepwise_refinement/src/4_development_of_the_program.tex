\section{Development of the Program}

We now formulate the stepwise generation of partial solutions to the 8-queens
problem by the following first version of a program:

\begin{lstlisting}
    variable board, pointer, safe;
    considerfirstcolumn;
    repeat trycolunm;
        if safe then
        begin setqueen; considernextcolumn
        end else regress
    until lastcoldone $\vee$ regressouttofirstcol
\end{lstlisting}


This program is composed of a set of more primitive instructions (or
procedures) whose actions may be described as follows:

\vspace{5mm}
\begin{small}
\emph{considerthiscolumn}.  The problem essentially consists of inspecting the
safety of squares.  A pointer variable designates the currently inspected
square.  The column in which this square lies is called the currently inspected
column.  This procedure initializes the pointer to denote the first column.

\emph{trycolumun}.  Starting at the current square of inspection in the
currently considered column, move down the column until a safe square is found,
in which case the boolean variable \emph{safe} is set to \textbf{true}, or
until the last square is reached and is also unsafe, in which case the variable
\emph{safe} is set to \textbf{false}.

\emph{setqueen}.  A queen is positioned into the last inspected square.

\emph{considernextcolumn}.  Advance to the next column and initialize its
pointer or inspection.

\emph{regress}.  Regress to a column where it is possible to move the
positioned queen further down, and remove the queens positioned in the columns
over which regression takes place.  (Note that we may have to regress over at
most two columns. Why?)
\end{small}
\vspace{5mm}

The next step of program development was chosen to refine the descriptions of
the instructions \emph{trycolumn} and \emph{regress} as follows:

\begin{lstlisting}
    procedure trycolumn;
    repeat advancepointer; testsquare
    until safe $\vee$ lastsquare

    procedure regress;
        begin reconsiderpriorcolumn
            if regressouttofirstcol then
            begin removequeen;
                if lastsquare then
                begin reconsiderpriorcolumn;
                if $\neg$ regressouttofirstcol then
                    removequeen
                end
            end
        end
\end{lstlisting}

The program is expressed in terms of the instructions:

\begin{quote}
\emph{considerfirstcolumn}\\
\emph{cibsidernextcolumn}\\
\emph{reconsiderpriorcolumn}\\
\emph{advancepointer}\\
\emph{testsquare} (sets the variable \emph{safe})\\
\emph{setqueen}\\
\emph{removequeen}\\
\end{quote}

and of the predicates:

\begin{quote}
\emph{lastsquare}\\
\emph{lastcoldone}\\
\emph{regressouttofirstcol}\\
\end{quote}

In order to refine these instructions and predicates further in the direction
of instructions and predicates available in common programming languages, it
becomes necessary to express them in terms of data representable in those
languages.  A decision on how to represent the relevant facts in terms of data
can therefore no longer be postponed.  First priority in decision making is
given to the problem of how to represent the positions of the queens and of the
square being currently inspected.

The most straightforward solution (i.e.\ the one most closely reflecting a
wooden chessboard occupied by marble pieces) is to introduce a Boolean square
matrix with $B[i, j] = \textbf{true}$ denoting that square $(i, j)$ is
occupied.  The success of an algorithm, however, depends almost always on a
suitable choice of its data representation in the light of the ease in which
this representation allows the necessary operations to be expressed.  Apart
from this, consideration regarding storage requirements may be of prime
importance (although hardly in this case).  A common difficulty in program
design lies in the unfortunate fact that at the stage where decisions about
data representations have to be made, it often is still difficult to foresee
the details of the necessary instructions operating on the data, and often
quite impossible to estimate the advantages of one possible representation over
another.  In general, it is therefore advisable to delay decisions about data
representation as long as possible (but not until it becomes obvious that no
realizable solution will suit the chosen algorithm).

In the problem presented here, it is fairly evident even at this stage that the
following choice is more suitable than a Boolean matrix in terms of simplicity
of later instructions as well as of storage economy.

$j$ is the index of the currently inspected column; $(x_j, j)$ is the
coordinate of the last inspected square; and the position of the queen in
column $k < j$ is given by the coordinate pair $(x_k, k)$ of the board.  Now
the variable declarations for pointer and board are refined into:

\begin{lstlisting}
    integer j  (0 $\leq$ j $\leq$ 9)
    integer array x[1:8] (0 $\leq$ x$_i$ $\leq$ 8)
\end{lstlisting}

and the further refinements of some of the above instructions and predicates
are expressed as:

\begin{lstlisting}
    procedure considerfirstcolumn;
        begin  j := 1; x[1] := 0 end

    procedure considernextcolumn;
        begin j := j + 1; x[j] := 0 end

    procedure reconsiderpriorcolumn;
        j := j - 1

    procedure advancepointer;
        x[j] := x[j] + 1

    Boolean procedure lastsquare;
        lastsquare := x[j] = 8;

    Boolean procedure lastcoldone;
        lastcoldone := j < 8

    Boolean procedure regressouttofirstcol;
        regressouttofirstcol := j < 1
\end{lstlisting}

At this stage, the program is expressed in terms of the instructions:

\begin{quote}
\emph{testsquare}\\
\emph{setqueen}\\
\emph{removequeen}\\
\end{quote}

As a matter of fact, the instructions \emph{setqueen} and \emph{removequeen}
may be regarded as vacuous, if we decide that the procedure \emph{testsquare}
is to determine the value of the variable \emph{safe} solely on the grounds of
the values $x_1 \ldots x_{j-1}$ which completely represent the positions of the
$j - 1$ queens so far on the board.  But unfortunately the instruction
\emph{testsquare} is the one most frequently executed, and it is therefore the
one instruction where considerations of efficiency are not only justified but
essential for a good solution of the problem.  Evidently a version of
\emph{testsquare} expressed only in terms of $x_1 \ldots x_{j-1}$ is
inefficient at best.  It should be obvious that \emph{testsquare} is executed
far more often than \emph{setqueen} and \emph{removequeen}.  The latter
procedures are executed whenever the column ($j$) is changed (say $m$ times),
the former whenever a move to the next square is undertaken (i.e.\ $x_j$ is
changed, say $n$ times).  However, \emph{setqueen} and \emph{removequeen} are
the only procedures which affect the chessboard.  Effeciency may therefore be
gained by the method of \emph{introducing auxiliary variables} $V(x_1 \ldots
x_j)$ such that:

\begin{enumerate}
    \item Whether a square is safe can be completed more easily from $V(x)$
          than from $x$ directly (say in $u$ units of computation instead of
          $ku$ units of computation).

    \item The computation of $V(x)$ from $x$ (whenever $x$ changes) is not too
          complicated (say of $v$ units of computation).
\end{enumerate}

The introduction of $V$ is advantageous (apart from considerations of storage
economy), if

\begin{quote}
$n(k - 1)u > mu$ \qquad or \qquad $\dfrac{n}{m}(k - 1) > \dfrac{v}{u}$,
\end{quote}

i.e.\ if the gain is greater than the loss in computation units.

A most straightforward solution to obtain a simple version of \emph{testsquare}
is to introduce a Boolean matrix $B$ such that $B[i,j] = \textbf{true}$
signifies that square $(i, j)$ is not taken by another queen.  But
unfortunately its recomputation whenever a new queen is removed ($v$) is
prohibitive (why?) and will more than outweigh the gain.

The realization that the relevant condition for safety of a square is that the
square must lie neither in a row nor in a diagonal already occupied by another
queen, leads to a much more economic choice of $V$.  We introduce Boolean
arrays \verb|a|, \verb|b|, \verb|c| with the meanings:

\begin{quote}
$a_k = \textbf{true}$ : no queen is positioned in row $k$\\
$b_k = \textbf{true}$ : no queen is positioned in the /-diagonal $k$\\
$c_k = \textbf{true}$ : no queen is positioned in the \textbackslash-diagonal $k$\\
\end{quote}

The choice of the index ranges of these arrays is made in view of the fact that
squares with equal sum of their coordinates lie on the same /-diagonal, and
those with equal difference lie on the same \textbackslash-diagonal.  With row
and column indices from 1 to 8, we obtain:

\begin{lstlisting}
    Boolean array a[1:8], b[2:16], c[-7:7]
\end{lstlisting}

Upon every introduction of auxiliary data, care has to he taken of their
\emph{correct initialization}.  Since our algorithm starts with an empty
chessboard, this fact must be represented by initially assigning the value
\textbf{true} to all components of the arrays \verb|a|, \verb|b|, and \verb|c|.
We can now write:

\begin{lstlisting}
procedure testsquare;
    safe := a[x[j]] $\wedge$ b[j + x[j]] $\wedge$ c[j - x[j]]

procedure setqueen;
    a[x[j]] := b[j + x[j]] := x[j - x[j]] := false

procedure removequeen;
    a[x[j]] := b[j + x[j]] := c[j - x[j]] := true
\end{lstlisting}

The correctness of the latter procedure is based on the fact that each queen
currently on the board had been positioned on a safe square, and that all
queens positioned after the one to he removed now had already been removed.
Thus the square to be vacated becomes safe again.

A critical examination of the program obtained so far reveals that the variable
\verb|x[j]| occurs very often, and in particular at those places of the program
which are also executed most often.  Moreover, examination of \verb|x[j]|
occurs much more frequently than reassignment of values to \verb|j|.  As a
consequence, the principle of introduction of new auxiliary data can again be
applied to increase efficiency: a new variable

\begin{lstlisting}
    integer i
\end{lstlisting}

is used to represent the value so far denoted by \verb|x[j]|.  Consequently
\verb|x[j] := i| must always be executed before \verb|j| is increased, and
\verb|i := x[j]| after \verb|j| is decreased.  This final step of program
development leads to the reformulation of some of the above procedures as
follows:

\begin{lstlisting}
procedure testsquare;
    safe := a[i] $\wedge$ b[i + j] $\wedge$ c[i - j]

procedure setqueen;
    a[i] := b[i + j] := c[i - j] := false

procedure removequeen;
    a[i] := b[i + j] := c[i - j] := true

procedure considerfirstcolumn;
    begin j := 1; i := 0 end

procedure advancepointer;
    i := i + 1;

procedure considernextcolumn;
    begin x[j] := i; j := j + 1; i := 0 end

Boolean procedure lastsquare;
    lastsquare := i = 8
\end{lstlisting}

The final program, using the procedures

\begin{quote}
\emph{testsquare}\\
\emph{setqueen}\\
\emph{regress}\\
\emph{removequeen}\\
\end{quote}

and with the other procedures directly substituted, now has the form

\begin{lstlisting}
    j := 1; i := 0;
    repeat
        repeat i :=  i + 1; testsquare
        until safe $\vee$ (i = 8);
        if safe then
        begin setqueen; x[j]  := i; j := j + 1; i := 0
        end else regress
    until (j > 8) $\vee$ (i < 1);
    if j > 8 then PRINT(x) else FAILURE
\end{lstlisting}

It is noteworthy that this program still displays the structure of the version
designed in the first step. Naturally other, equally valid solutions can be
suggested and be developed by the same method of stepwise program refinement.
It is particularly essential to demonstrate this fact to students.  One
alternative solution was suggested to the author by E.~W.~Dijkstra.  It is
based on the view that the problem consists of a stepwise extension of the
board by one column containing a safely positioned queen, starting with a
null-board and terminating with 8 columns. The process of extending the board
is formulated as a procedure, and the natural method to obtain a complete board
is by \emph{recursion} of this procedure. It can easily be composed of the same
set of more primitive instructions which were used in the first solution.

\begin{lstlisting}
    procedure Trycolumn(j);
        begin integer i; i := 0;
        repeat i := i + 1; testsquare;
            if safe then
            begin setqueen; x[j] := i; 
                if j < 8 then Trycolumn(j+1); 
                if $\neg$ safe then removequeen 
            end 
        until safe $\vee$ (i = 8)
    end
\end{lstlisting}

The program using this procedure then is

\begin{lstlisting}
    Trycolumn(1); 
    if safe then PRINT(x) else FAILURE
\end{lstlisting}

(Note that due to the introduction of the variable \verb|i| local to the
recursive procedure, every column has its own pointer of inspection \verb|i|.
As a consequence, the procedures

\begin{quote}
\emph{testsquare}\\
\emph{setqueen}\\
\emph{removequeen }\\
\end{quote}

must be declared locally within \verb|Trycolumn| too, because they refer to the
\verb|i| designating the scanned square in the \emph{current} column.)
