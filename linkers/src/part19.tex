\section{Part 19}

I've pretty much run out of linker topics. Unless I think of something new, ll
I'make tomorrow's post be the last one, for a total of 20.

\texttt{\_\_start} and \texttt{\_\_stop} Symbols

A quick note about another GNU linker extension. If the linker sees a section
in the output file which can be part of a C variable nam--the name contains
only alphanumeric characters or underscore--the linker will automatically
define symbols marking the start and stop of the section. Note that this is not
true of most section names, as by convention most section names start with a
period. But the name of a section can be any string; it doesn't have to start
with a period. And when that happens for section NAME, the GNU linker will
define the symbols \texttt{\_\_start\_NAME} and \texttt{\_\_stop\_NAME} to the
address of the beginning and the end of section, respectively.

This is convenient for collecting some information in several different object
files, and then referring to it in the code. For example, the GNU C library
uses this to keep a list of functions which may be called to free memory. The
\texttt{\_\_start} and \texttt{\_\_stop} symbols are used to walk through the
list.

In C code, these symbols should be declared as something like \texttt{extern
char \_\_start\_NAME[]}. For an extern array the value of the symbol and the
value of the variable are the same.

\subsection{Byte Swapping}

The new linker I am working on, gold, is written in C++. One of the attractions
was to use template specialization to do efficient byte swapping. Any linker
which can be used in a cross-compiler needs to be able to swap bytes when
writing them out, in order to generate code for a big-endian system while
running on a little-endian system, or vice-versa. The GNU linker always stores
data into memory a byte at a time, which is unnecessary for a native linker.
Measurements from a few years ago showed that this took about 5\% of the
linker's CPU time. Since the native linker is by far the most common case, it
is worth avoiding this penalty.

In C++, this can be done using templates and template specialization. The
idea is to write a template for writing out the data. Then provide two
specializations of the template, one for a linker of the same endianness
and one for a linker of the opposite endianness. Then pick the one to use
at compile time. The code looks this; I'm only showing the 16-bit case for
simplicity.

\begin{lstlisting}[language=C++]
    // Endian simply indicates whether the host is big endian or not.

    struct Endian
    {
        public:
        // Used for template specializations.
        static const bool host_big_endian = __BYTE_ORDER == __BIG_ENDIAN;
    };

    // Valtype_base is a template based on size (8, 16, 32, 64) which
    // defines the type Valtype as the unsigned integer of the specified
    // size.

    template
    struct Valtype_base;

    template<>
    struct Valtype_base<16>
    {
        typedef uint16_t Valtype;
    };

    // Convert_endian is a template based on size and on whether the host
    // and target have the same endianness.  It defines the type Valtype
    // as Valtype_base does, and also defines a function convert_host
    // which takes an argument of type Valtype and returns the same value,
    // but swapped if the host and target have different endianness.

    template
    struct Convert_endian;

    template
    struct Convert_endian
    {
        typedef typename Valtype_base::Valtype Valtype;

        static inline Valtype
        convert_host(Valtype v) { return v; }
    };

    template<>
    struct Convert_endian<16, false>
    {
        typedef Valtype_base<16>::Valtype Valtype;

        static inline Valtype
        convert_host(Valtype v) { return bswap_16(v); }
    };

    // Convert is a template based on size and on whether the target is
    // big endian.  It defines Valtype and convert_host like
    // Convert_endian.  That is, it is just like Convert_endian except in
    // the meaning of the second template parameter.

    template
    struct Convert
    {
        typedef typename Valtype_base::Valtype Valtype;

        static inline Valtype
        convert_host(Valtype v)
        {
            return Convert_endian
            ::convert_host(v);
        }
    };

    // Swap is a template based on size and on whether the target is big
    // endian.  It defines the type Valtype and the functions readval and
    // writeval.  The functions read and write values of the appropriate
    // size out of buffers, swapping them if necessary.

    template
    struct Swap
    {
        typedef typename Valtype_base::Valtype Valtype;

        static inline Valtype
        readval(const Valtype* wv) { return Convert::convert_host(*wv); }

        static inline void
        writeval(Valtype* wv, Valtype v) { *wv = Convert::convert_host(v); }
    };
\end{lstlisting}

Now, for example, the linker reads a 16-bit big-endian value using
\texttt{Swap<16,true>::readval}.  This works because the linker always knows
how much data to swap in, and it always knows whether it is reading big- or
little-endian data.
