\section{Part 17}

\subsection{Warning Symbols}

The GNU linker supports a weird extension to ELF used to issue warnings when
symbols are referenced at link time. This was originally implemented for a.out
using a special symbol type. For ELF, I implemented it using a special section
name.

If you create a section named \texttt{.gnu.warning.SYMBOL}, then if and when
the linker sees an undefined reference to \texttt{SYMBOL}, it will issue a
warning. The warning is triggered by seeing an undefined symbol with the right
name in an object file. Unlike the warning about an undefined symbol, it is not
triggered by seeing a relocation entry. The text of the warning is simply the
contents of the \texttt{.gnu.warning.SYMBOL} section.

The GNU C library uses this feature to warn about references to symbols like
\texttt{gets} which are required by standards but are generally considered to
be unsafe. This is done by creating a section named \texttt{.gnu.warning.gets}
in the same object file which defines \texttt{gets}.

The GNU linker also supports another type of warning, triggered by sections
named \texttt{.gnu.warning} (without the symbol name). If an object file
with a section of that name is included in the link, the linker will issue
a warning. Again, the text of the warning is simply the contents of the
\texttt{.gnu.warning} section. I don't know if anybody actually uses this
feature.
