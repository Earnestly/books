\section{Part 14}

\subsection{Link Time Optimization}

I've already mentioned some optimizations which are peculiar to the linker:
relaxation and garbage collection of unwanted sections. There is another
class of optimizations which occur at link time, but are really related to
the compiler. The general name for these optimizations is \emph{link time
optimization} or \emph{whole program optimization}.

The general idea is that the compiler optimization passes are run at link
time. The advantage of running them at link time is that the compiler can then
see the entire program. This permits the compiler to perform optimizations
which can not be done when sources files are compiled separately. The most
obvious such optimization is inlining functions across source files. Another
is optimizing the calling sequence for simple function---e.g., passing more
parameters in registers, or knowing that the function will not clobber all
registers; this can only be done when the compiler can see all callers of
the function. Experience shows that these and other optimizations can bring
significant performance benefits.

Generally these optimizations are implemented by having the compiler write
a version of its intermediate representation into the object file, or into
some parallel file. The intermediate representation will be the parsed
version of the source file, and may already have had some local optimizations
applied. Sometimes the object file contains only the compiler intermediate
representation, sometimes it also contains the usual object code. In the former
case link time optimization is required, in the latter case it is optional.

I know of two typical ways to implement link time optimization. The first
approach is for the compiler to provide a pre-linker. The pre-linker examines
the object files looking for stored intermediate representation. When it finds
some, it runs the link time optimization passes. The second approach is for
the linker proper to call back into the compiler when it finds intermediate
representation. This is generally done via some sort of plugin API\@.

Although these optimizations happen at link time, they are not part of the
linker proper, at least not as I defined it. When the compiler reads the stored
intermediate representation, it will eventually generate an object file, one
way or another. The linker proper will then process that object file as usual.
These optimizations should be thought of as part of the compiler.

\subsection{Initialization Code}

C++ permits globals variables to have constructors and destructors. The
global constructors must be run before \texttt{main} starts, and the global
destructors must be run after \texttt{exit} is called. Making this work
requires the compiler and the linker to cooperate.

The a.out object file format is rarely used these days, but the GNU a.out
linker has an interesting extension. In a.out symbols have a one byte type
field. This encodes a bunch of debugging information, and also the section
in which the symbol is defined. The a.out object file format only supports
three sections–text, data, and bss. Four symbol types are defined as sets:
text set, data set, bss set, and absolute set. A symbol with a set type is
permitted to be defined multiple times. The GNU linker will not give a multiple
definition error, but will instead build a table with all the values of the
symbol. The table will start with one word holding the number of entries, and
will end with a zero word. In the output file the set symbol will be defined as
the address of the start of the table.

For each C++ global constructor, the compiler would generate a symbol named
\texttt{\_\_CTOR\_LIST\_\_} with the text set type. The value of the symbol
in the object file would be the global constructor function. The linker
would gather together all the \texttt{\_\_CTOR\_LIST\_\_} functions into
a table. The startup code supplied by the compiler would walk down the
\texttt{\_\_CTOR\_LIST\_\_} table and call each function. Global destructors
were handled similarly, with the name \texttt{\_\_DTOR\_LIST\_\_}.

Anyhow, so much for a.out. In ELF, global constructors are handled in a
fairly similar way, but without using magic symbol types. I'll describe what
gcc does. An object file which defines a global constructor will include a
\texttt{.ctors} section. The compiler will arrange to link special object files
at the very start and very end of the link. The one at the start of the link
will define a symbol for the \texttt{.ctors} section; that symbol will wind
up at the start of the section. The one at the end of the link will define a
symbol for the end of the \texttt{.ctors} section. The compiler startup code
will walk between the two symbols, calling the constructors. Global destructors
work similarly, in a \texttt{.dtors} section.

ELF shared libraries work similarly. When the dynamic linker loads a shared
library, it will call the function at the \texttt{DT\_INIT} tag if there is
one. By convention the ELF program linker will set this to the function named
\texttt{\_init}, if there is one. Similarly the \texttt{DT\_FINI} tag is called
when a shared library is unloaded, and the program linker will set this to the
function named \texttt{\_fini}.

As I mentioned earlier, three are also \texttt{DT\_INIT\_ARRAY},
\texttt{DT\_PREINIT\_ARRAY}, and \texttt{DT\_FINI\_ARRAY} tags, which are set
based on the \texttt{SHT\_INIT\_ARRAY}, \texttt{SHT\_PREINIT\_ARRAY}, and
\texttt{SHT\_FINI\_ARRAY} section types. This is a newer approach in ELF, and
does not require relying on special symbol names.
