\section{Part 12}

I apologize for the pause in posts. We moved over the weekend. Last Friday
at\&t told me that the new DSL was working at our new house. However, it did
not actually start working outside the house until Wednesday. Then a problem
with the internal wiring meant that it was not working inside the house until
today. I am now finally back online at home.

\subsection{Symbol Resolution}

I find that symbol resolution is one of the trickier aspects of a linker.
Symbol resolution is what the linker does the second and subsequent times
that it sees a particular symbol. I've already touched on the topic in a few
previous entries, but let's look at it in a bit more depth.

Some symbols are local to a specific object files. We can ignore these for
the purposes of symbol resolution, as by definition the linker will never
see them more than once. In ELF these are the symbols with a binding of
\texttt{STB\_LOCAL}.

In general, symbols are resolved by name: every symbol with the same name is
the same entity. We've already seen a few exceptions to that general rule.
A symbol can have a version: two symbols with the same name but different
versions are different symbols. A symbol can have non-default visibility: a
symbol with hidden visibility in one shared library is not the same as a symbol
with the same name in a different shared library.

The characteristics of a symbol which matter for resolution are:

\begin{itemize}
    \item The symbol name

    \item The symbol version.

    \item Whether the symbol is the default version or not.

    \item Whether the symbol is a definition or a reference or a common symbol.

    \item The symbol visibility.

    \item Whether the symbol is weak or strong (i.e., non-weak).

    \item Whether the symbol is defined in a regular object file being included
          in the output, or in a shared library.

    \item Whether the symbol is thread local.

    \item Whether the symbol refers to a function or a variable.
\end{itemize}

The goal of symbol resolution is to determine the final value of the symbol.
After all symbols are resolved, we should know the specific object file or
shared library which defines the symbol, and we should know the symbol's type,
size, etc. It is possible that some symbols will remain undefined after all
the symbol tables have been read; in general this is only an error if some
relocation refers to that symbol.

At this point I'd like to present a simple algorithm for symbol resolution, but
I don't think I can. I'll try to hit all the high points, though. Let's assume
that we have two symbols with the same name. Let's call the symbol we saw first
A and the new symbol B. (I'm going to ignore symbol visibility in the algorithm
below; the effects of visibility should be obvious, I hope.)

\begin{enumerate}
    \item If A has a version:

        \begin{itemize}
            \item If B has a version different from A, they are actually
                  different symbols.

            \item If B has the same version as A, they are the same symbol;
                  carry on.

            \item If B does not have a version, and A is the default version of
                  the symbol, they are the same symbol; carry on.
            \item Otherwise B is probably a different symbol.  But note that if
                  A and B are both undefined references, then it is possible
                  that A refers to the default version of the symbol but we
                  don't yet know that.  In that case, if B does not have a
                  version, A and B really are the same symbol.  We can't tell
                  until we see the actual definition.
        \end{itemize}

    \item If A does not have a version:

        \begin{itemize}
            \item If B does not have a version, they are the same symbol; carry
                  on.

            \item If B has a version, and it is the default version, they are
                  the same symbol; carry on.

            \item Otherwise, B is probably a different symbol, as above.
        \end{itemize}

    \item If A is thread local and B is not, or vice-versa, then we have an
          error.

    \item If A is an undefined reference:

        \begin{itemize}
            \item If B is an undefined reference, then we can complete the
                  resolution, and more or less ignore B.

            \item If B is a definition or a common symbol, then we can resolve
                  A to B.
        \end{itemize}

    \item If A is a strong definition in an object file:

        \begin{itemize}
            \item If B is an undefined reference, then we resolve B to A.

            \item If B is a strong definition in an object file, then we have a
                  multiple definition error.

            \item If B is a weak definition in an object file, then A overrides
                  B.  In effect, B is ignored.

            \item If B is a common symbol, then we treat B as an undefined
                  reference.

            \item If B is a definition in a shared library, then A overrides B.
                  The dynamic linker will change all references to B in the
                  shared library to refer to A instead.
        \end{itemize}

    \item If A is a weak definition in an object file, we act just like the
          strong definition case, with one exception: if B is a strong
          definition in an object file.  In the original SVR4 linker, this case
          was treated as a multiple definition error.  In the Solaris and GNU
          linkers, this case is handled by letting B override A.

    \item If A is a common symbol in an object file:

        \begin{itemize}
            \item If B is a common symbol, we set the size of A to be the
                  maximum of the size of A and the size of B, and then treat B
                  as an undefined reference.

            \item If B is a definition in a shared library with function type,
                  then A overrides B (this oddball case is required to
                  correctly handle some Unix system libraries).

            \item Otherwise, we treat A as an undefined reference.
        \end{itemize}

    \item If A is a definition in a shared library, then if B is a definition
          in a regular object (strong or weak), it overrides A.  Otherwise we
          act as though A were defined in an object file.

    \item If A is a common symbol in a shared library, we have a funny case.
          Symbols in shared libraries must have addresses, so they can't
          be common in the same sense as symbols in an object file. But
          ELF does permit symbols in a shared library to have the type
          \texttt{STT\_COMMON} (this is a relatively recent addition). For
          purposes of symbol resolution, if A is a common symbol in a shared
          library, we still treat it as a definition, unless B is also a common
          symbol. In the latter case, B overrides A, and the size of B is set
          to the maximum of the size of A and the size of B.
\end{enumerate}

I hope I got all that right.
