\section{Part 8}
\subsection{ELF Segments}

Earlier I said that executable file formats were normally the same as object
file formats.  That is true for ELF, but with a twist.  In ELF, object files
are composed of sections: all the data in the file is accessed via the section
table.  Executables and shared libraries normally contain a section table,
which is used by programs like \texttt{nm}.  But the operating system and the
dynamic linker do not use the section table.  Instead, they use the segment
table, which provides an alternative view of the file.

All the contents of an ELF executable or shared library which are to be loaded
into memory are contained within a segment (an object file does not have
segments).  A segment has a type, some flags, a file offset, a virtual address,
a physical address, a file size, a memory size, and an alignment.  The file
offset points to a contiguous set of bytes which are the contents of the
segment, the bytes to load into memory.  When the operating system or the
dynamic linker loads a file, it will do so by walking through the segments and
loading them into memory (typically by using the \texttt{mmap} system call).  All
the information needed by the dynamic linker–the dynamic relocations, the
dynamic symbol table, etc.--are accessed via information stored in special
segments.

Although an ELF executable or shared library does not, strictly speaking,
require any sections, they normally do have them.  The contents of a loadable
section will fall entirely within a single segment.

The program linker reads sections from the input object files.  It sorts and
concatenates them into sections in the output file.  It maps all the loadable
sections into segments in the output file.  It lays out the section contents in
the output file segments respecting alignment and access requirements, so that
the segments may be mapped directly into memory.  The sections are mapped to
segments based on the access requirements: normally all the read-only sections
are mapped to one segment and all the writable sections are mapped to another
segment.  The address of the latter segment will be set so that it starts on a
separate page in memory, permitting \texttt{mmap} to set different permissions on
the mapped pages.

The segment flags are a bitmask which define access requirements.  The defined
flags are \texttt{PF\_R}, \texttt{PF\_W}, and \texttt{PF\_X}, which mean, respectively,
that the contents must be made readable, writable, or executable.

The segment virtual address is the memory address at which the segment contents
are loaded at runtime.  The physical address is officially undefined, but is
often used as the load address when using a system which does not use virtual
memory.  The file size is the size of the contents in the file.  The memory
size may be larger than the file size when the segment contains uninitialized
data; the extra bytes will be filled with zeroes.  The alignment of the segment
is mainly informative, as the address is already specified.

The ELF segment types are as follows:

\begin{itemize}
    \item \texttt{PT\_NULL}: A null entry in the segment table, which is ignored.

    \item \texttt{PT\_LOAD}: A loadable entry in the segment table.  The operating
          system or dynamic linker load all segments of this type.  All other
          segments with contents will have their contents contained completely
          within a \texttt{PT\_LOAD} segment.

    \item \texttt{PT\_DYNAMIC}: The dynamic segment.  This points to a series of
          dynamic tags which the dynamic linker uses to find the dynamic symbol
          table, dynamic relocations, and other information that it needs.

    \item \texttt{PT\_INTERP}: The interpreter segment.  This appears in an
          executable.  The operating system uses it to find the name of the
          dynamic linker to run for the executable.  Normally all executables
          will have the same interpreter name, but on some operating systems
          different interpreters are used in different emulation modes.

    \item \texttt{PT\_NOTE}: A note segment.  This contains system dependent
          note information which may be used by the operating system or the
          dynamic linker.  On GNU/Linux systems shared libraries often have a
          ABI tag note which may be used to specify the minimum version of the
          kernel which is required for the shared library.  The dynamic linker
          uses this when selecting among different shared libraries.

    \item \texttt{PT\_SHLIB}: This is not used as far as I know.

    \item \texttt{PT\_PHDR}: This indicates the address and size of the segment
          table.  This is not too useful in practice as you have to have
          already found the segment table before you can find this segment.

    \item \texttt{PT\_TLS}: The TLS segment.  This holds the initial values for
          TLS variables.

    \item \texttt{PT\_GNU\_EH\_FRAME} (\texttt{0x6474e550}): A GNU extension used to
          hold a sorted table of unwind information.  This table is built by
          the GNU program linker.  It is used by gcc's support library to
          quickly find the appropriate handler for an exception, without
          requiring exception frames to be registered when the program start.  

    \item \texttt{PT\_GNU\_STACK} (\texttt{0x6474e551}): A GNU extension used to
          indicate whether the stack should be executable.  This segment has no
          contents.  The dynamic linker sets the permission of the stack in
          memory to the permissions of this segment.

    \item \texttt{PT\_GNU\_RELRO} (\texttt{0x6474e552}): A GNU extension which tells
          the dynamic linker to set the given address and size to be read-only
          after applying dynamic relocations.  This is used for const variables
          which require dynamic relocations.

\end{itemize}

\subsection{ELF Sections}

Now that we've done segments, lets take a quick look at the details of ELF
sections.  ELF sections are more complicated than segments, in that there are
more types of sections.  Every ELF object file, and most ELF executables and
shared libraries, have a table of sections.  The first entry in the table,
section 0, is always a null section.

ELF sections have several fields.

\begin{itemize}
    \item Name.

    \item Type.  I discuss section types below.

    \item Flags.  I discuss section flags below.

    \item Address.  This is the address of the section.  In an object file this
          is normally zero.  In an executable or shared library it is the
          virtual address.  Since executables are normally accessed via
          segments, this is essentially documentation.

    \item File offset.  This is the offset of the contents within the file.

    \item Size.  The size of the section.

    \item Link.  Depending on the section type, this may hold the index of
          another section in the section table.

    \item Info.  The meaning of this field depends on the section type.

    \item Address alignment.  This is the required alignment of the section.
          The program linker uses this when laying out the section in memory.

    \item Entry size.  For sections which hold an array of data, this is the
          size of one data element.
\end{itemize}

These are the types of ELF sections which the program linker may see.

\begin{itemize}

    \item \texttt{SHT\_NULL}: A null section.  Sections with this type may be
          ignored.

    \item \texttt{SHT\_PROGBITS}: A section holding bits of the program.  This is
          an ordinary section with contents.

    \item \texttt{SHT\_SYMTAB}: The symbol table.  This section actually holds the
          symbol table itself.  The section contents are an array of ELF symbol
          structures.

    \item \texttt{SHT\_STRTAB}: A string table.  This type of section holds
          null-terminated strings.  Sections of this type are used for the
          names of the symbols and the names of the sections themselves.

    \item \texttt{SHT\_RELA}: A relocation table.  The link field holds the index
          of the section to which these relocations apply.  These relocations
          include addends.

    \item \texttt{SHT\_HASH}: A hash table used by the dynamic linker to speed
          symbol lookup.

    \item \texttt{SHT\_DYNAMIC}: The dynamic tags used by the dynamic linker.
          Normally the \texttt{PT\_DYNAMIC} segment and the \texttt{SHT\_DYNAMIC}
          section will point to the same contents.

    \item \texttt{SHT\_NOTE}: A note section.  This is used in system dependent
          ways.  A loadable \texttt{SHT\_NOTE} section will become a
          \texttt{PT\_NOTE} segment.

    \item \texttt{SHT\_NOBITS}: A section which takes up memory space but has no
          associated contents.  This is used for zero-initialized data.

    \item \texttt{SHT\_REL}: A relocation table, like \texttt{SHT\_RELA} but the
          relocations have no addends.

    \item \texttt{SHT\_SHLIB}: This is not used as far as I know.

    \item \texttt{SHT\_DYNSYM}: The dynamic symbol table.  Normally the
          \texttt{DT\_SYMTAB} dynamic tag will point to the same contents as this
          section (I haven't discussed dynamic tags yet, though).

    \item \texttt{SHT\_INIT\_ARRAY}: This section holds a table of function
          addresses which should each be called at program startup time, or,
          for a shared library, when the library is opened by dlopen.

    \item \texttt{SHT\_FINI\_ARRAY}: Like \texttt{SHT\_INIT\_ARRAY}, but called at
          program exit time or dlclose time.

    \item \texttt{SHT\_PREINIT\_ARRAY}: Like \texttt{SHT\_INIT\_ARRAY}, but called
          before any shared libraries are initialized.  Normally shared
          libraries initializers are run before the executable initializers.
          This section type may only be linked into an executable, not into a
          shared library.

    \item \texttt{SHT\_GROUP}: This is used to group related sections together, so
          that the program linker may discard them as a unit when appropriate.
          Sections of this type may only appear in object files.  The contents
          of this type of section are a flag word followed by a series of
          section indices.

    \item \texttt{SHT\_SYMTAB\_SHNDX}: ELF symbol table entries only provide a
          16-bit field for the section index.  For a file with more than 65536
          sections, a section of this type is created.  It holds one 32-bit
          word for each symbol.  If a symbol's section index is
          \texttt{SHN\_XINDEX}, the real section index may be found by looking in
          the \texttt{SHT\_SYMTAB\_SHNDX} section.

    \item \texttt{SHT\_GNU\_LIBLIST} (\texttt{0x6ffffff7}): A GNU extension used by
          the prelinker to hold a list of libraries found by the prelinker.

    \item \texttt{SHT\_GNU\_verdef} (\texttt{0x6ffffffd}): A Sun and GNU extension
          used to hold version definitions (I'll take about symbol versions at
          some point).

    \item \texttt{SHT\_GNU\_verneed} (\texttt{0x6ffffffe}): A Sun and GNU extension
          used to hold versions required from other shared libraries.

    \item \texttt{SHT\_GNU\_versym} (\texttt{0x6fffffff}): A Sun and GNU extension
          used to hold the versions for each symbol.
\end{itemize}

These are the types of section flags.

\begin{itemize}
    \item \texttt{SHF\_WRITE}: Section contains writable data.

    \item \texttt{SHF\_ALLOC}: Section contains data which should be part of the
          loaded program image.  For example, this would normally be set for a
          \texttt{SHT\_PROGBITS} section and not set for a \texttt{SHT\_SYMTAB}
          section.

    \item \texttt{SHF\_EXECINSTR}: Section contains executable instructions.

    \item \texttt{SHF\_MERGE}: Section contains constants which the program linker
          may merge together to save space.  The compiler can use this type of
          section for read-only data whose address is unimportant.

    \item \texttt{SHF\_STRINGS}: In conjunction with \texttt{SHF\_MERGE}, this
          means that the section holds null terminated string constants which
          may be merged.

    \item \texttt{SHF\_INFO\_LINK}: This flag indicates that the info field in the
          section holds a section index.

    \item \texttt{SHF\_LINK\_ORDER}: This flag tells the program linker that when
          it combines sections, this section must appear in the same relative
          order as the section in the link field.  This can be used to ensure
          that address tables are built in the expected order.

    \item \texttt{SHF\_OS\_NONCONFORMING}: If the program linker sees a section
          with this flag, and does not understand the type or all other flags,
          then it must issue an error.

    \item \texttt{SHF\_GROUP}: This section appears in a group (see
          \texttt{SHT\_GROUP}, above).

    \item \texttt{SHF\_TLS}: This section holds TLS data.
\end{itemize}
